\documentclass[10pt]{report}
\usepackage[usenames]{color}
\usepackage{amssymb} 
\usepackage{amsmath} 
\usepackage[utf8]{inputenc}
\usepackage{listings}
\usepackage{multirow}

\title{MA Draft}
\author{Ema Skottova, M21a}

\begin{document}
\maketitle

\tableofcontents

\chapter{Introduction}
In the first round of the Swiss Olympiad in Informatics there is always a creativity task, which does not have an optimal solution, but there are many approaches to reach an acceptable solution. In the fall of 2019, the task was to optimize a train network. During the first round I only submitted a very simple solution, which worked for a special case of small networks. The goal of this paper is to compare the ability of two optimization methods to solve this task. The chosen methods are simulated annealing and genetic algorithms, both of which will be explained in the theory chapter. I have first encountered the former at a programming camp and the latter was suggested by my supervisor.


\chapter{Theoretical Background}

\section{Optimization Problems}
In an optimization problem the objective is to find the best (optimal) solution. Example problems are finding the minimum/maximum of a mathematical function, the shortest train connection between two cities, a way to maximize the profit of a company, or the optimal shape for a vehicle. Mathematically a solution can be modelled using a set of parameters, which clearly define a solution. A value function is needed to compare different solutions. In applied mathematics or outside of mathematics, the term often refers to problems where it is difficult to find the optimal solution. Therefore, instead of looking for the optimal solution,the objective becomes finding a sufficiently good one. In this project that type of problem is considered.

\section{Optimization Methods}
An optimization method is a procedure to find the optimal (or a sufficiently good) solution. 

For this project the  methods simulated annealing and genetic algorithms were chosen. They will be compared in their ability to optimize a train network. 

\subsection{Simulated Annealing}
The general idea is that at the beginning, the program will take larger steps in testing sets of parameter values and with time its step size decreases, and it only looks at options closer to options that were already good. %TODO: Add pseudocode

\subsection{Genetic Algorithms}
A genetic algorithm starts with a few ‘individuals’ (sets of parameter values) and the individuals multiply with ‘mutations’ (changes in parameter values) and after a few mutations only the best few survive. This process continues until a sufficiently good answer is found. %TODO: Add pseudocode


\chapter{Complete Problem Statement}

As already stated in the introduction, the goal of this paper is to compare two optimization methods. In order to do this, one first needs a problem to optimize. For this paper, the creativity task SOIway of the first round of the Swiss Olympiad in Informatics 2019/2020 was chosen. The task is as follows:

The program is given constants and a list of events, then it has to print actions.

Constants:
\begin{itemize}
    \item minimal amount of time to change a line
    \item maximal number of passengers allowed on a train
    \item maximal number of passengers allowed at a station
\end{itemize}


Events (with their time of appearance):
\begin{itemize}
    \item Appearance of a station (coordinates of the station and the type of the station)
    \item Appearance of a passenger (starting station and type of end station)
    \item Additional train
    \item Additional train line
\end{itemize}

Actions:
\begin{itemize}
    \item Setting/Changing a line (stations that the line will cover)
    \item Adding a train to a line (line and the station where it starts)
    \item Boarding a passenger
    \item Unboarding a passenger
\end{itemize}

The program has failed if it produces invalid output. If the program does not fail, the process ends if
(a)	All passengers have arrived at their final destination
or
(b)	There are more passengers than allowed at a station


\chapter{Methods}

A general solution, which depends on a set of parameters, is written first. Then, the two optimization methods will be used to find the best set of parameters.


\section{General Solution}
The general solution goes through each event and action and evaluates whether it should make an action. An action will be taken if the sum of the parameters which influence it is above 100.

\subsection{Parameters}

\subsubsection{Picking up passenger}
\begin{itemize}
    \item Distance (time, stops, line changes in current train)
    \item Number of passengers on the train
    \item Number of passengers at the station
    \item Network capacity
    \item These values for the following train(s)/passenger(s)/station(s)
\end{itemize}

\subsubsection{Letting passenger leave}
\begin{itemize}
    \item Correct station
    \item Number of passengers at the station
    \item Number of passengers on the train
    \item Distance (time, stops, line changes) of current line
    \item Distance (time, stops, line changes) of connecting lines
    \item Capacity of current line
    \item Capacity of neighboring lines
    \item Values for other trains/stations/passengers
\end{itemize}

\subsubsection{Changing line for train}
\begin{itemize}
    \item Passengers per train on each line
\end{itemize}

\subsubsection{Changing train line (time)}
\begin{itemize}
    \item Balance in current network
    \item Important trains on current line
\end{itemize}

\subsubsection{New train line (route)}
\begin{itemize}
    \item Frequency of visits
    \item Number of passengers starting
\end{itemize}

\section{Evaluation}
The two methods have to be able to be compared. The solutions which are generated by the two methods will be compared depending on whether the system overloads and how many time steps the solutions take to either finish or overload the system. If these two are equal, both solutions are equally good. As the main objective is to distribute all passengers without overloading the system, a solution which achieves this is strictly better than a solution which does not, no matter how many time steps they run. If this goal can be achieved by both solutions, the solution which can do this fewer time steps is better. If the system becomes overloaded in both solutions, the solution which runs longer without overloading wins. The following chart shows the evaluation in a more compact way.

\begin{tabular}{|c|c|c|c|c|}
    \hline
    \multicolumn{2}{|c|}{Method 1} & \multicolumn{2}{|c|}{Method 2} & \multirow{2}{*}{Result}                         \\ \hline
    Reason for end                 & Steps                          & Reason for end          & Steps &               \\ \hline
    (b)                            & $x$                            & (b)                     & $< x$ & Method 1 wins \\ \hline
    (b)                            & $x$                            & (b)                     & $x$   & Draw          \\ \hline
    (b)                            & $x$                            & (a)                     & $y$   & Method 2 wins \\ \hline
    (a)                            & $x$                            & (a)                     & $>x$  & Method 1 wins \\ \hline
    (a)                            & $x$                            & (a)                     & $x$   & Draw          \\ \hline
\end{tabular}


\chapter{Process}
\section{Input Generator}
At first a program which generates random test data was written. This program’s parameters were set to match a test data sample from the original competition.

\section{Input/Output}
Next, the first part of the main program was created. This part is responsible for reading the input and printing the output.

\section{General Solution V1}
After this, the general program which calculates a solution will be written. This version will only consider very simple parameters, which can be calculated easily. The goal of this version is to have a running program so that tests can be done on the program. In particular, this should make it possible to test first versions of the two methods.

\section{Simulated Annealing}
This algorithm will be implemented and tested using the first version of the general solution.

\section{General Solution V2}
After first tests with simulated annealing, improvements will be added to the general solution.

\section{Genetic Algorithm}
This algorithm will be implemented and tested. First comparisons between the algorithms can be made.

\section{Improvements General Solution}
In subsequent versions, missing parameters will be added and other necessary improvements will be made.

\chapter{Issues}
\section{General Solution}
While implementing the general solution, larger issues arose for the first time. The main problem was that it is difficult to test parts of the general solution before a primitive version of it is running. This lead to over 400 lines of code being written, where the only testing was compiling it a few times. 

\subsection{Circular References}
Due to some circular references the program had to be restructured a few times. At first, solving some of the references worked by just changing the order of the definition of some classes. Later, it became necessary to make a forward declaration of all the classes. This wasn't enough either, and after a lot of googling I found out how to do a so-called inline definition, which means one can write a method outside the class. This enabled first defining the classes and their methods and only later actually writing out the methods.

\subsection{Class hierarchy, Pointers and Priority Sorting}
In order to fix an issue with calling methods of derived classes I was forced to use pointers (which I hadn't done very often before that). With this I ran into a new issue as I needed to sort the events the pointers were pointing at by time. As it is impossible to overwrite comparison operators for pointers in C++ a lambda function had to be used. 

\chapter{Program}

In this chapter, the most important methods and classes will be explained to show the essence of the program while excluding uninteresting details.

\section{General Information}
The program was written in C++. Currently its length is greater than 650 lines, including comments and blank lines. 

\section{Structure}

\subsection{Object-oriented programming}
The program is object-oriented. The main idea behind object-oriented programming (OOP) is to create independent classes, each with its own properites and methods (functions). Every class is only responsible for a part of the program. This has some advantages: One of the main advantages of OOP is that similar pieces of code can be written only once and then reused easily. Another advantage is that using OOP, a program is easier to debug as it is always clear which part does what. In addition to helping maintain a large project it also simplifies collaboration, as every person only has to worry about the class they are working on and some of the interactions with other classes. This eliminates the need to know anything about the rest of the project and avoids naming conflicts (i.e. naming two different variables which do something completely different in different parts of the program $x$). 

\subsection{$Solution$}
$Solution$ is the class for a general solution. When initialized, it takes a set of parameters automatically computes the number of time steps and whether it succeeds in getting all passengers to their target station.

Pseudocode for calculation of general solution: 
\lstset {language=C++}
\begin{lstlisting}
    events = input_events
    overfilled = false
    while(!overfilled)
    {   
        e = next_event
        events.remove(next_event)
        overfilled = e.run(events)
    }
\end{lstlisting}

The most important part here happens in the $run()$ methond, which is different for each event. This will be described separately for each type of event:

\subsubsection{NewObject}
The appearance of an object (train, station, passenger, line) is an event. Here, $run$ mainly initializes the object and adds it to necessary lists. For lines and trains, the line or train is added to the system immediately after this.

\subsubsection{TrainArrival}
When a train arrives at a station, multiple things happen. Firstly the train is registered at the station. Next, it is decided which passengers should leave the train and which passengers should board. It never makes sense for a passenger to stay on a train if it has the option to leave the train to a station of the correct type (i.e. the type of station where the passenger was headed). Therefore, this will be done automatically, no matter what the parameters for that action say. The same thing happens with passengers who should board or leave a train if that action leads to an overfilled train or station.

\subsubsection{TrainDeparture}
The most important thing that happens when a train departs is that a new train arrival and departure are scheduled, i.e. added to the list of events.


\subsection{$Station, Passenger, Train, Line$}
The clear objects in the problem of course also get their own classes. All these classes have properties such as their id and time of appearance. 

\subsection{Optimization methods}
The optimization methods have been implemented as described in the theory chapter, where their pseudocode can be found.


\chapter{Results}


\end{document}
